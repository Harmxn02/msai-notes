\documentclass[a4paper, 12pt]{article}

% Packages
\usepackage[utf8]{inputenc}
\usepackage{amsmath, amssymb, amsthm}
\usepackage{graphicx}
\usepackage{hyperref}
\usepackage[margin=2.5cm]{geometry}

\usepackage{charter}

% Document information
\title{Mathematics for Machine Learning \textemdash{} LE 3}
\author{Harman Preet Singh}

% Custom theorem environments
\newtheorem{theorem}{Theorem}
\newtheorem{lemma}[theorem]{Lemma}
\newtheorem{proposition}[theorem]{Proposition}
\newtheorem{corollary}[theorem]{Corollary}
\newtheorem{definition}{Definition}
\newtheorem{example}{Example}

\begin{document}
% chktex-file 44

\maketitle
\tableofcontents

\section{Wat is een limiet van een functie?}
\section{Standaardlimieten}
\section{Rekenregels}
\section{Oneindige limieten}

\section{Grote-O-notatie}
\subsection{Definitie van grote O}
\subsection{Rekenregels voor grote O}
\subsection{Berekenen van limieten met grote O}

\section{Informaticatoepassingen}
\subsection{Grote-O-notatie bij het bepalen van complexiteit}
\subsection{Limietgedrag van functies die de groei van het internet beschrijven}
\subsection{Betrouwbaarheidsfuncties}





\end{document}