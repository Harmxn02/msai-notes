\documentclass[a4paper, 12pt]{article}

% Packages
\usepackage[utf8]{inputenc}
\usepackage{amsmath, amssymb, amsthm}
\usepackage{graphicx}
\usepackage{hyperref}
\usepackage[margin=2.5cm]{geometry}

\usepackage{charter}

% Document information
\title{Mathematics for Machine Learning \textemdash{} LE 6}
\author{Harman Preet Singh}

% Custom theorem environments
\newtheorem{theorem}{Theorem}
\newtheorem{lemma}[theorem]{Lemma}
\newtheorem{proposition}[theorem]{Proposition}
\newtheorem{corollary}[theorem]{Corollary}
\newtheorem{definition}{Definition}
\newtheorem{example}{Example}

\begin{document}
% chktex-file 44

\maketitle
\tableofcontents

\section{Integraal als oppervlakte}
\section{De functie als afgeleide van zijn oppervlaktefunctie}
\section{Primitieven}
\section{Bepaalde en onbepaalde integraal}
\section{Oneigenlijke integralen}

\section{Informaticatoepassingen}
\subsection{Schatten van sommen met behulp van integraalrekening en de complexiteit van Quicksort}
\subsection{Defuzzificatie}
\subsection{Weergave van kleuren}




\end{document}