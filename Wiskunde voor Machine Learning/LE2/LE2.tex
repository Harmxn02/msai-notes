\documentclass[a4paper, 12pt]{article}

% Packages
\usepackage[utf8]{inputenc}
\usepackage{amsmath, amssymb, amsthm}
\usepackage{graphicx}
\usepackage{hyperref}
\usepackage[margin=2.5cm]{geometry}

\usepackage{charter}

% Document information
\title{Mathematics for Machine Learning \textemdash{} LE 2}
\author{Harman Preet Singh}

% Custom theorem environments
\newtheorem{theorem}{Theorem}
\newtheorem{lemma}[theorem]{Lemma}
\newtheorem{proposition}[theorem]{Proposition}
\newtheorem{corollary}[theorem]{Corollary}
\newtheorem{definition}{Definition}
\newtheorem{example}{Example}

\begin{document}
% chktex-file 44

\maketitle
\tableofcontents

\section{Reeksen}
\subsection{Somnotatie}
\subsection{Reeksen als rijen van partiële sommen}

\section{Meetkundige reeksen, convergentie en divergentie}
\section{Rekenregels en andere reeksen}

\section{Toepassingen}
\subsection{De complexiteit van Quicksort}
\subsection{Exponential smoothing}
\subsection{Annuïteiten}


\end{document}